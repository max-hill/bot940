\documentclass{article}
% math packages
\usepackage{amsmath,amssymb,amsthm,amsfonts,bbm,breqn,tikz,graphicx}
\usepackage[margin=1in]{geometry} \graphicspath{ {./images/} }
% colorful notes
\usepackage{color} \definecolor{red}{rgb}{1,0,0} \definecolor{blue}{rgb}{0,0,1}
\def\red{\color{red}} \def\blue{\color{blue}} \newcommand{\rnote}[1]{{\red #1}}
\newcommand{\bnote}[1]{{\blue #1 }} % usage: \rnote{foo} and \bnote{foo}
% math environments
\newtheorem{theorem}{theorem} \newtheorem{definition}{definition}
\newtheorem{lemma}{lemma} \newtheorem{remark}{remark}
\newtheorem{prop}{proposition} \newtheorem{corollary}{corollary}
\newtheorem{example}{example} \newtheorem{question}{question}
\newtheorem{claim}{claim} \newtheorem{problem}{problem}
% custom math commands
\newcommand{\vt}{\vskip 5mm} % vertical space
\newcommand{\fl}{\noindent\textbf} % first line
\def\[{\left [} \def\]{\right ]} \def\({\left (} \def\){\right )}
\def\<{\langle} \def\>{\rangle} % brackets and parentheses
\def\r{\mathbb{r}} \def\z{\mathbb{z}} \def\e{\mathbb{e}} \def\p{\mathbb{p}}
% vector macro
\usepackage{stackengine}
\newcommand\cv[1]{\setstackeol{,,}\bracketvectorstack{#1}^{\top}}
\newcommand\rv[1]{\setstackeol{,,}\bracketvectorstack{#1}} %usage: \cv{1,2,3}
% macros specific to this class
\newcommand{\advar}{\sigma_{a}^{2}} % additive variance
\newcommand{\bvar}{\sigma_{b}^{2}}  % between population variance
\newcommand{\mvar}{\sigma_{m}^{2}}  % mutational variance
\newcommand{\wvar}{\sigma_{w}^{2}}  % phenotypic variation within a population
\newcommand{\svar}{\sigma_{s}^{2}}  % per generation selection variance
\usepackage{dsfont}
% title
\title{botany 940 notes}
\begin{document}
\maketitle

\section{2022-02-01: brownian motion}
[harmon chapters 3-5]

\subsection{hamon chapter 3: intro to brownian motion}
we consider traits that have continuous distribution --- such as body mass. we
can model these traits with brownian motion. it is incorrect to equate brownian
motion models with models of pure genetic drift. (why?)

when modeling evolution using brownian motion, we usually are considering the
dynamics of the \textbf{mean character value} $\overline{z}$ in a population.
this is the average value of the trait within the population. we denote by
$\overline{z}(t)$ the value at time $t$. we can model this by a brownian motion
process. the mean is $\overline{z}(0)$ and the variance, or ``evolutionary rate
parameter'' is denoted by $\sigma^{2}$. we can simulate change under the
brownian motion model by drawing from normal distributions; simply put the
changes in a trait value over an interval of time $t$ is drawn from a normal
random variable with mean 0 and variance $\sigma^{2}t$.


what does this mean: ``we assume that mutations are drawn at random from a
distribution with mean 0 and mutational variance $\sigma_{m}^{2}$.''?
\begin{definition}[types of variance]
types of variance under this model:
\begin{itemize}
  \item $\mvar$ is the \textbf{mutational variance} the variance of the number of
    new distinct mutations added per generations (i think)
  \item $\bvar$ is \textbf{between-population phenotypic variance}, meaning the
    variance of mean trait values across many independent runs of evolutionary
    change over a certain time period.
  \item $\advar$ is the \textbf{additive genetic variance} within each
    population at some time $t$. additive genetic variance measures the total
    amount of genetic variation that acts additively (i.e. the contributions of
    each allele add together to predict the final phenotype)
  \item $\wvar$ is the total phenotypic variation within a population, including
    both non-additive genetic effects and environmental effects.
\end{itemize}
\end{definition}

\subsection{additive variance equilibrium}
\begin{claim}
  we have $\e\[ \advar (t)\] \approx 2n_{e}\sigma_{m}^{2}$ for large $t$.
\end{claim}
\begin{proof}
  over time, $\advar$ will change. this is due to genetic drift (which tends to
  decrease $\advar$) and mutational input (which tends to increase $\advar$). the
  expected value of $\advar$ changes from one generation to the next according to
  the formula:
  \begin{equation*} \e\[\advar(t+1) \]= \left( 1-\frac{1}{2n_{e}}
    \right)\e\[\sigma_{a}^{2}(t) \]+\sigma_{m}^{2}
  \end{equation*}
  where $t$ is the time in generations. the first term represents the loss due to
  drift. the second term represents the gain due to new mutations each generation.
  by induction, we have:

  \begin{equation}\label{eq:2}
    \e\[\advar(t) \] = \left( 1-\frac{1}{2n_{e}} \right)^{t}\left[ \advar(0) -2n_{e}\mvar \right] + 2n_{e}\mvar 
  \end{equation}
  where of course $\advar(0)$ is the starting value at time zero. this formula
  allows us to calculate the expected additive genetic variance at any time
  provided we know $\advar(0)$. since $\left( 1-\frac{1}{2n_{e}} \right)^{t}\to 0$
  as $t\to\infty$, we expect $\e\[ \advar (t) \]$ to approach equilibrium value of
  $2n_{e}\sigma_{m}^{2}$.
\end{proof}


\subsection{derivation of $\bvar(t)$}
  

assume $\advar$ is at equilibrium and thus constant. harmon claims this is
justified by equation \eqref{eq:2}, even though that equation is about the
expected value of $\advar$. then skipping some calculations,
\begin{equation}
  \label{eq:1}
  \bvar(t)= \frac{t\advar(t)}{n_{e}}
\end{equation}

substituting the equilibrium value $2n_{e}\mvar$ in for $\advar(t)$ gives
\begin{equation}
  \label{eq:4}
  \bvar(t) = 2t \mvar
\end{equation}

this says that the variation between two diverging populations depends on twice
the time since they diverged and the rate of mutational input. no dependence on
population size or starting state is observed, so under this model, long-term
rates of evolution are dominated by the supply of new mutations to a population.

actually \eqref{eq:4} is very general, holding under a range of models.
unfortunately, we cannot measure $\mvar$ in any natural population.




\begin{definition}[heritability]
  \textbf{heritability} is the proportion of total phenotypic variation $\wvar$
  within a population that is due to additive genetic effects; namely,
  \begin{equation}
    \label{eq:5}
    h^{2}=\frac{\advar}{\wvar}.
  \end{equation}
\end{definition}

by the additive variance equilibrium equation \eqref{eq:2}, we have
\begin{equation}
  \label{eq:6}
  h^{2}= \frac{2n_{e}\mvar}{\wvar}
\end{equation}
or equivalently,
\begin{equation}
  \label{eq:7}
  \mvar = \frac{h^{2}\wvar}{2n_{e}}
\end{equation}
and therefore by \eqref{eq:4},
\begin{equation}
  \label{eq:8}
  \bvar(t) = \frac{h^{2}\wvar t}{n_{e}}.
\end{equation}


\subsection{brownian motion under selection}
recall we with to model the path followed by population mean trait values under
mutation, selection, and drift. the brownian motion model applies even to some
cases with selection.

\subsubsection{random directional selection}
we assume directional selection, but with the strength and direction of
selection varying randomly from one generation to the next. each generation, we
model selection as being drawn from a normal distribution with mean 0 and
variance $\svar$.

in this case, for some reason, we have
\begin{equation}
  \label{eq:9}
  \bvar = \left( \frac{h^{2}\wvar}{n_{e}}+ \svar \right)t
\end{equation}
in this case, if variation in selection is much greater than varition due to
drive, then for some reason we have
\begin{equation}
  \label{eq:3}
  \bvar \approx \svar
\end{equation}

\subsubsection{random stabilizing selection}

on the other hand, if the trait is under stabilizing selection for a particular
optimal value, where the position of the optimal value changes randomly
according to a brownian motion process with variance $\sigma_{e}^{2}$, then
\begin{equation*}
  \bvar \approx \sigma^{2}_{e}
\end{equation*}

under both of these models, the pattern of trait evolution thorugh time still
follows a brownian motion model, even though the changes are dominated by
selection rather than drift. therefore brownian motion evolution does not assume
that characters are not under selection.

\subsubsection{constant directional selection}

not going to go through the equations on this one. if a trait is under
directional selection of strength $s$ in both populations, then we get
\begin{equation}
  \label{eq:10}
  \bvar(t) = \frac{h^{2}\wvar t}{n_{e}}
\end{equation}
this does not depend on $s$ and in fact is the same equation as \eqref{eq:8}
which was derived earlier. thus we can't tell by looking at two living
populations whether the trait we are measuring was subject to selection, or
something like that.

\subsection{brownian motion on a tree}
i am mostly familiar with this section from the diffusion group. a new thing is
the \textbf{phylogenetic variance-covariance matrix} \textbf{c}. this matrix has a special
structure. for phylogenetic trees with $n$ species, this is an $n\times n$
matrix, with each row and column corresponding to one of the $n$ taxa in the tree.
along the diagonal are the total distances of each taxon from the root of the
tree, while the off-diagonal elements are the total branch lengths shared by
particular pairs of taxa. so it is a matrix $(\sigma^{2}x_{ij})$ where $x_{ii}$
is the length of the path $p(i,r)$ from leaf $i$ to the root, and $x_{ij}$ is
the length of those edges shared by paths $p(i,r)$ and $p(j,r)$.

example: suppose we have a phylogenetic tree with two leaves $1$ and $2$ with
lengths $t_{2},t_{3}$ respectively, and a root edge of length $t_{1}$. we can
describe the trait values for the two species as a single draw from a
multivariate normal distribution. each trait has the same expected value
$\overline{z}(0)$, which is the starting value of the trait at the root, and two
traits have the following covariance matrix:

\begin{equation*}
  \begin{bmatrix}
    \sigma^2 (t_1 + t_2) & \sigma^2 t_1 \\
    \sigma^2 t_1 & \sigma^2 (t_1 + t_3) \\
\end{bmatrix}
= \sigma^2
\begin{bmatrix}
    t_1 + t_2 & t_1 \\
    t_1 & t_1 + t_3 \\
\end{bmatrix} = \sigma^2 \mathbf{c}
\end{equation*}
we see what $\mathbf{c}$ is in this case.

\section{2022-02-01: fitting brownian motion models to single characters}
[harmon chp 4]
\subsection{log transforms}
if your data is a continuous trait and you think percentage changes matter more
than absolute changes, then you should do a log-transform of your data.

\subsection{estimating rates using independent contrasts}

\textbf{phylogenetical idependent contrasts} (pics) are a way to estimate the
rate of character change across a phylogeny. there is an algorithm, the
algorithm of independent contrats, from felsenstein. it is used to estimate the
rate of evolution under a brownian model. it uses some of the following ideas.
the \textbf{raw contrast} is the difference between the value of the character
at two chosen tips: $c_{ij} = x_{i}-x_{j}$. under the brownian model, $c_{ij}$
has expectation zero and variance proportional to $v_{i}+v_{j}$. the
\textbf{standardized contrast} is $s_{ij} = \frac{c_{ij}}{v_{i}+v_{j}}$. it
follows that $s_{ij}\sim n(0,\sigma^{2})$.

\subsection{estimating rates using maximum likelihood}
another way to estimate evolutionary rate is by finding the maximum-likelihood
parameter values for a brownian motional model. our data is character values at
the tips of the tree which we assume was generated with the brownian model.
there is a standard formula for the likelihood of drawing from a multivariate
normal distribution; remember we wish to find the parameter values that maximize
this function; this is done using optimization algorithms.

actually, in some cases we have analytic solutions.

\textbf{restrictied maximum liklihood} (reml) refers to an approach which uses
ml after transforming the data set to remove nuisance paramters, such as the
root state.


\subsection{bayesian approach to evolutionary rates}
this approach uses explicit priors for parameter values, and then runs an mcmc
to estimate posterior distributions of parameter estimates.


\subsection{fitting brownian models to multiple characters}
motivation: a wide variety of hypotheses can be framed as tests of correlations
between continuously varying traits across species. for example, is the body
size of a species related to its metabolic rate? how does the head length of a
species relate to overall size, and do deviations from this relationship relate
to an animal’s diet?

we make a distinction between \textbf{standard correlation} and
\textbf{evolutionary correlation}. the first means that one trait predicts the
value of another. the second is when two traits evolve together due to a process
like mutation, drift, or selection. if there is an evolutionary correlation
between two characters, that means we can predict the magnitude and direction of
changes in one character given knowledge of evolutionary changes in another.

phylogenetic relatedness alone can lead to a relationship between two variables
that are not in face evolving together. in that case, there is a confounding
variable -- the clade the related species belong to.

\subsection{modeling the evolution of correlated characters}
multivariate brownian motion. each trait evolves under a brownian motion, but
these brownian motions are not necessarily independent, as specified by some
covariance matrix $\mathbf{r}$.

recall that correlations at the tips depend also on branch lengths, specified by
some matrix $c$ (see chapter 3). then the variance-covariance matrix is the
kroeneker product $\mathbf{r}\otimes \mathbf{c}$, an $nr\times nr$ matrix, where
$n$ is the number of leaves of the tree and $r$ the number of traits.

here again, there is a formula for the likelihood of observing data
$\mathbf{x}_{nr}$ given the input parameters. also, we can estimate these
parameters analytically with formulas similar to the univariate case.

\subsection{testing for evolutionary correlations}
fit a model with correlation. fit a model without correlation. calculate the
optimal log-likelihood in both cases. apply log-likelihood test.

there is also a bayesian approach that i didn't read.

\subsection{testing with traditional approaches (pic, pgls)}

pgls is \textbf{phylogenetic general least squares}. in general least squares,
we construct a model of the relationship between the column vectors
$x,y\in \mathbb{R}^{n}$ of trait values and the whose correlation we wish to test:
\begin{equation*}
  y = x_{d}b + \epsilon
\end{equation*}
where
$x_{d} = \begin{bmatrix} 1 & x_1 \\ 1 & x_2 \\ \vdots & \vdots \\ 1 & x_n
  \\ \end{bmatrix}$, $b$ is $2\times 1$, and the error vector $\epsilon$ is
assumed to be multivariate normal with mean zero and some covariance matrix
$\omega$. in the brownian model, we assume that the residuals have variances and
covariances following the structure of the phylogenetic tree: that is, given by
the matrix $\mathbf{c}$. we carry out standard least square to estimate model
parameters. the first term in $b$ is the phylogenetic mean $\overline{z}(0)$,
and the second term is an estimate for the slope of the relationship between $y$
and $x$.

another way to think about a pgls model is that we are treating $x$ as a fixed
property of species. the deviation of $y$ from what is predicted by $x$ is what
evolves under a brownian motion model.
\section{2022-02-01: fitting brownian motion models ot multiple characters}
[harmon chapter 5]

many hypotheses can be framed as tests of correlations between continuously varying traits across species. for example is the body size of a species related to the metabolic rate? multivariate gaussians. kroenecker product of the $\mathbf{r}$ and $\mathbf{c}$ matrices. too much typing. didn't take better notes.

\section{2022-02-08: continuous traits: ou models}
[harmon chapter 6, butler et al 2004]

\subsection{beyond brownian motion}
[harmon chapter 6]

this chapter considesr ways in which comparative methods can move beyond simple
brownian motion models:
\begin{enumerate}
  \item by transforming the variance-covariance matrix
  \item by incorporating variation in rates of evolution
  \item by accounting for evolutionary constrains
  \item by modeling adaptive radiation and ecological opportunity.
\end{enumerate}

\subsubsection{transforming the covariance matrix}
three statistical models to test whether data deviates from a constant-rate
process evolving on a phylogenetic tree. these involve three kinds of pagel tree
transformations, each acting on the covariance matrix $\mathbf{c}$. they are $\lambda,\delta,\kappa$.
\begin{itemize}
  \item for pagel's $\lambda$, you just multiply all the off-diagonal entries of
  \textbf{c} by $\lambda\in [0,1]$. this is like shrinking all the internal
  branches by a factor of $\lambda$ but leaving the leaf edges unchanged.
  \item pagel's $\delta$ is for capturing variation in rates of evolution over time. it
  is done by raising all the entries of $\mathbf{c}$ to the power of $\delta$.
  depending on whether $\delta>1$ or $\delta<1$, this has the effect of stretching
  or compressing the node heights, with deep and shallow branches affected
  differently.
  \item pagel's $\kappa$ transformation is used to capture patterns of speciational
  change in trees. in this, all the branch lengths are raised to the power of
  $\kappa\geq 0$. this transformation isn't really used anymore.
\end{itemize}
\subsubsection{variation in rates of trait evolution across clades}
brownian motion assumes that the rate of change $\sigma^{2}$ is constant. there
are three ways to test for differences in rate of evolution across clades:
\begin{enumerate}
\item compare the magnitude of independent contrasts across clades. the key idea
  is that when rates are high, we expect to see large independent contrasts in
  that part of the tree; in particular, the square of an independent contrast is
  an estimate of the brownian motion rate parameter $\sigma^{2}$.
\item use model comparison approaches to compare the fit of single- and
  multipe-rate modelds to data on trees
\item use a bayesian approach combined with reversible-jump machinery to try to
  drift the places on the tree where the rate shifts have occurred.
\end{enumerate}
\bnote{i didn't really understand the last two.}


\subsubsection{non-brownian evolution under stabilizing selection}
ornstein-uhlenbeck (ou) model.

\subsubsection{early burst models}
test for adaptive radiations by looking for bursts of trait evolution deepin the
tree. the simplest method is to use a time-inhomogenous brownian motion model;
that is, a model in which the rate parameter $\sigma^{2}$ varies through time, such as
\begin{equation*}
  \sigma^{2}(t) = \sigma_0^{2}e^{bt}, \quad(b<0)
\end{equation*}
this model also generates multivariate normal distributions of tip values. in
particular, given model parameters $(\overline{\zeta}_0,\sigma^{2},b)$, we can
obtain a vector of means and a covariance matrix for the tips. from this, we can
use the multivariate gaussian distribution function to calculate a likelihood,
which can then be used in an ml or bayesian statistical framework.

\subsubsection{peak shift models}
punctuated trait shifts. has to do with multivariate models.


\subsection{phylogenetic comparative analysis: modeling adaptive evolution}
[butler and king 2004]

ou model incorporates both selection and drift. this paper develops the method for one quantitative character. this allows us to test hypotheses regarding adaptation in different selective regimes against data using ml. 

if two species are in different selective regimes, we expect their mean phenotypes to differ. but in brownian models, all lineages share the same expected mean phenotype. also we should expect variance in mean phenotypes to remain bounded, but in brownian motion models, this variance grows without bound.

ou model includes brownian motion as a special case. consider the evolution of a quantitative character $x$ along the branch of a phylogenetic tree. we can decompose the change in $x$ into deterministic and stochastic parts:
\begin{equation*}
  dx(t) = \alpha \left[ \theta -x(t) \right]dt + \sigma db(t)
\end{equation*}
where $\alpha$ measures the strength of selection and $\sigma$ measures the intensity of the random fluctuations in the evolutionary process. the parameter $\theta$ gives the optimum trait value.

there are two evolutionary interpretations:
\begin{itemize}
\item $x(t)$ represents the species' mean phenotype, with $\theta$ taking the role of a local maximum in a fitness landscape, 
\item something else
\end{itemize}
each lineage in the tree evolves according to its own ou process, which requires
that there be one optimum per branch of the phylogeny.



\section{uyeda (2013) a novel bayesian method for inferring...'}

\subsection{abstract and intro}
devolops reversible-jump bayesian method of fitting multi-optima ou models to
phylogenetic comparative data. goal: estimate the placement and magnitude of
adaptive shifts from the data.

adaptive landscape. microevolution assumes it's fixed, but this is not
reasonable for macroevolution.

software often requires optima to be ``painted'' onto branches according to
researcher's hypotheses.

\subsection{methods}

ou parameters
\begin{align*} d\overline{z} &= \alpha(\theta-\overline{z})+\sigma dw
\end{align*} here $\alpha$ is the \textit{rate of adaptation}, measured in
inverse time units; $\theta$ is the optimum value of the process; $\sigma^{2}$
is the per unit time magnitude of uncorrelated diffusion (measured in units of
squared trait units per time unit).

the phylogenetic half-life is $\alpha^{-1}\log (2)$, which is measured in time
units. this is an easier way to interpret $\alpha$. it is the amount of time it
takes for the expected trait value to get halfway to the phenotypic optimum. if
this is small (i.e. $\alpha$ is large) then covariance between species erodes
exponentially fast, so that the results will resemble a bm process. otoh, if
$\alpha$ is small (ie much shorted than the youngest split on a phylogeny) then
the model will resemble a white noise process (i.e., residual trait values are
completely uncorrelated). [i don't understand that].

a useful compound parameter for ou process is: $v_{y}=\sigma^{2}/2\alpha$, which
is \textbf{stationary variance}, the equilibrium variance of an ou process
evolving around a stationary optimum $\theta$.

\textbf{multi-optima ou models} are a modification of standard ou model in which
adaptive optimum $\theta$ is allowed to vary accross the phylogeny according to
discrete shifts in adaptive regimes. however, the number of shifts and their
locations on the phylogeny are not fixed. instead the algorithm presented in
this paper estimates those things.

\subsection{reversible-jump model} start with a fully resolved phylogeny with
$n$ taxa. the optima at the root is $\theta_0$, and the vector of optima on the
tree is $\overline{\theta}=(\theta_0,\ldots,\theta_{k})$ where $k$ is the number
of shifts between adaptive optima. the locations of shifts are mapped onto the
phylogeny as locations $\mathbf{l}=\left\{ l_{1},\ldots,l_{k} \right\}$. given
these parameters, the distribution of tip states $y$ is multivariate normal with
expectation
\begin{equation*} \e\[y|y_{0},\alpha,\mathbf{l},\overline{\theta} \]=
\mathbf{w}\overline{\theta}
\end{equation*} where $y_0$ is the root state and \textbf{w} a matrix of weights
used to calculate the weighted average of adaptive optima, discounted by an
exponentially decreasing function that depends on the rate parameter $\alpha$
and the time since the species evolved under a given optima. the full
explanation and derivation is not given in this paper.

the covariance matrix for $y$ is given by
\begin{align*} y_{i} &= \frac{\sigma^{2}}{2\alpha}\left[ 1-e^{-2\alpha t_{0,i}}
\right]\\ y_{ij} &= e^{-\alpha t_{ij}}\frac{\sigma^{2}}{2\alpha}\left[
1-e^{-2\alpha t_{0,ij}} \right]
\end{align*} where $t_{0,i}$ is the time from the root to species $i$; $t_{ij}$
is the total time separating species $i$ and $j$; and $t_{0,ij}$ is the total
time separating the root and the mrca of species $i$ and $j$.

what is a pruning algorithm (``we used a pruning algorithm to speed computation
and calculate conditional likelihood'' p904)?

the reversible-jump framework uses metropolis-hastings algorithm to explore
models with varying dimensionality through the course of the mcmc.
dimensionality here refers to the number of optima shifts which occur. the
amount of time the mcmc spends in a given model is proportional to its posterior
probability.

adaptive optima are constructed in a way that makes them unique (i.e. different
from the other optima). this was done because it's easier for the detailed
balance condition to be shown to be true in that case.


\subsection{simulation study}
variation in body sizes in 226 extant chelonians (turtles and tortoises).

i am a little confused about what uyeda and harmon (2013) are doing in the
simulation study of chelonian body sizes, and have tried to write out the basic
steps of their approach. if i am understanding correctly, the basic procedure is
as follows:

1. the input for the simulation study is a phylogenetic species tree t with
the 226 extant chelonians as leaves, along with (a) the average body size of the
taxa at each leaf of t, and (b) some prior distribution on the space of
possible parameters $k, \mathbf{l}, \theta, \alpha$, and $\sigma^{2}$.

2. the mcmc algorithm jumps around the space of possible combinations of
parameters; the markov chain is constructed such that that its stationary
distribution is the posterior probability. therefore in the long run, the
proportion of time the markov chain spends in a particular state converges to
the posterior probability of that state. after running the markov chain for
500,000 or so steps, a sufficent degree of convergence has taken place, and a
sample is taken.

3. repeat step 2 several thousand times obtain several thousand samples.

4. the sample distribution is taken to be our estimate of the posterior
distribution. the posterior distribution is a distribution *on the space of
parameters*.

5. provided that chelonian evolution is well approximated by the ou process, the
posterior distribution is thought to say something informative about where/when
optima shifts occurred, intensity of selection, etc (e.g. figure 6)

is this a correct understanding of the basic procedure?


\section{omeara et al 2006: testing for different rates of continuous trait evolution using likelihood}
\subsection{basic properties of character evolution on trees}

\textbf{disparity} is commonly measured as variance of the states of the taxa. so higher disparity for a character means the taxa are less similar for the character. bm is the standard model for continuous character evolution. of course disparity depends on both variance of bm (the ``rate of character evolution'') and the underlying phylogenetic tree (i.e. divergence times).

suppose we have a tree with $n$ leaves. let $c$ be the $n\times n$ matrix whose $ij$-th entry is the distance from the root to the mrca of leaves $i$ and $j$. then
\begin{equation*}
  \e\[ \mathrm{disparity}\]= \sigma^{2}\left[ \frac{1}{n} \mathrm{tr}(c) - \frac{1}{n^{2}} \mathbf{1}^{\top}c \mathbf{1} \right]
\end{equation*}


note that extinctions can reduce the age of mrca of a clade. the ml estimator of the rate of phenotypic evolution under bm is
\begin{equation*}
  \widehat{\sigma}^{2} =  \frac{\left[ x-\e\[x \]  \right]^{\top} c^{-1}\left[ x-\e\[x \]  \right]}{n}
\end{equation*}
where $x$ is the column vector of tip observations.


the \textbf{noncensored approach} is where each branch is painted with a rate parameter $\sigma^2$. there is a ml estimator for this in which the rates are allowed to vary on different places of the tree. i think harmon went over this and it's in my notes above. the \textbf{censored apprach} makes no assumptions about where or how the change has happened somewhere on that branch. 


\section{2022-02-20: discrete character evolution}
\subsection{hamon chapter 7: models of discrete character evolution}
snakes are lizards that lost their limbs.

Mk Model: analogue to JC model. For evolution of discrete charactesr that have a
set number of fixed states. The Extended Mk Model does sth similar

\section{2022-02-21: pagel and meade 2006}
this paper is about bayesian analysis of correlated evolution of discrete
characters by reversible-jump markov mcmc.

synopsis: the rj markov chain visits these models in proportion to their
posterior probabilities, thereby directly estimating the support for the
hypothesis of correlated evolution. in addition, the rj markov chain
simultaneously estimates the posterior distributions of the rate parameters of
the model of trait evolution.

they implement the method by analyziong the question of whether mating system
and advertisement of estrus in females have coevolved in the old world monkeys
and great apes. in this case, there are two binary traits
\begin{itemize}
\item advertisement of estrus
\item mating system (multimale mating or not)
\end{itemize} 
which yields four possible states: 00,01,10,11. if the traits have evolved
independently, then the rate of change between the two states of one variable
will not depend on the state of the other. only states which share exactly one
component can transition to each other (e.g. 00 can transition to 10 or 01 but
not to 11). there are 8 possible transitions, and the rates for a given
transition may or may not equal the rate of a different transition. there are
4140 distinct models wherin each rate is associated with anywhere from 1 to 8
possible rate classes. for example, (1,1,1,1,1,1,1,1) is one possible model, in
which all of the rates are the same (i.e. are in rate class 1); under
(1,1,1,2,2,2,2,2), the first three transitions share the same rate which is
different from the last five; under (1,2,3,4,5,6,7,8) the transition rates are
all different.

if a markov chain following this acceptance rule is allowed to run for long
enough, it reaches a stationary distribution. at stationarity, a properly
constructed chain wanders through the universe of possible states, visiting
better and worse models in proportion to their posterior probabilities, rather
than inexorably moving toward better outcomes, as an optimizing approach, such
as maximum likelihood, would do. the stationary distribution of the chain
thereby simultaneously samples the posterior distribution of models of trait
evolution, the parameters of these models, and the likelihood of the data.
  
they describe a number of moves on the space $\left\{ (x_{1},\ldots,x_{8}):
x_{i}\in [8] \right\}$ of rate models. note that 15 of the 4140 rate models
correspond to independent evolution.

they use the ``bayes factors'' (bf) test, defined as
\begin{equation*} \mathrm{bf}_{ij}=\frac{\p\[ d|m_{i} \]}{\p\[ d|m_{j} \]}
\end{equation*} where $i,j$ are models. it's just the ratio of marginal
likelihoods. if $\mathrm{bf}_{ij}>1$ then the average likelihood of the data
under model $i$ is greater than that for model $j$. bayes factors can be
estimated with mcmc. there is some criteria for determining if a bf is
sufficiently large to be strong or very strong evidence of correlated evolution.


In addition, 99.86\% of the models at convergence included at least one rate in
the zero bin. BF of about 28 gives strong evidence of correlated evolution.

Figure 7 is helpful. Interpreted adaptively, mating system changes first in
evolution, and this selects for a change in the female display strategy. The
alternative hypothesis that females alter their displaying strategy and that
this selects for a multimale mating system is not supported.


\subsection{Question} I once (naively) implemented a Metropolis-Hastings MCMC
algorithm in an attempt to uniformly sample paths from one corner of a hypercube
to another. As the dimension of hypercube increased, it would tend to get
``stuck,'' endlessly rejecting proposed moves. This limited the ability to
explore the sample space.

Are there general criteria for predicting whether MCMC is likely to be
effective for a given sample space?

\section{2022-02-28: Ives and Garland 2009: Phylogenetic Logistic Regression for
Binary Dependent Variables}

This paper develops statistical methods for phylogenetic logistic regression in
which the dependent variable is binary and not independent between species. That
is, species more closely related are more likely to have the same character
value. These methods give a way to estimate the phylogenetic signal of binary
traits.

Takeaway: phylogenetic logistic regression should be used rather than standard
logistic regression.


\textbf{Phylogenetic signal} is the tendancy for related sepcies to resemble
each other. Multispecies datasets (e.g. for the relation of brain size to body
size) should be analyzed with methods that account for this. A tree is
\textbf{hierarchical} if it is not a star tree. Under Brownian Motion assumption
of trait evolution, there will be some amount of phylogenetic signal. But this
resemblance between species violates one or more assumptions of most common
statistical methods, ie independence.

If the only concern is hypothesis testing, then it is possible to apply
conventional statistical methods (eg ANOVA), apply F ratios, and compare those
test statistics with null distributions that have been derived by simulated or
randomizing data in accordance with a specified phylogenetic tree and assumed
model of evolution.

The method of phylogenetically independent constrasts (PIC) assumes a known
phylogenetic tree and an assumed a model of character evolution can be used to
transform tip data to make them have equal expected variances and remove
correlations related to phylogenetic signal. \bnote{Not really understand PIC.}
PIC is a special case of phylogenetic least squares methods. Under Brownian
motion model of character evolution, PGLS and PIC calculations give the same
parameter estimates and statistical tests.

For this paper, there is a binary dependent variable $Y$ and zero, one or more
independent variables $X$, which can be either continuous or discrete. The
approach involves generalized linear models (GLMs) that can be used to analyze
data from any exponential family of distributions. This method does not require
apriori assignment of phylogenetic signal. However this means that the 
statistical tests will not be the same since there is additional uncertainty
from this estimation in the model.  

The model presented in this paper gives a plausible model that can be used for
statistical analyses. There are two steps (multivariate case): First, values of
$Y$ evolve up the phylogenetic tree with asymptotic probability of meing in
state 1 equal to $\mu$ and transition rate $\alpha$. This gives a $Y$ value of 0
or 1 to each species. Second, the value of $Y_{i}$ for each speices evolves
toward either 0 or 1 dependig on the values of the independent variables $X^{j}$
and the regression coefficients $b_{j}$. No assumption is made about the
distribution of independent variables, or even that they contain phylogenetic
signal.

Output: estimates of regression coefficients that account for phylogenetic
correlations. An estimate of the strength of phylogenetic signal in the residual
variation.


I am still trying to understand the multivariate case in the Ives and Garland
paper. The process is divided into two components. In the first component, a
binary character evolves along the phylogenetic tree according to some Markov
process starting at the root and having transition rates
$\alpha=(\alpha_{0},\alpha_{1})$ and stationary distribution $\mu$ which is some
function of these rates. This yeilds a vector $Y$ of binary states corresponding
to the leaves of the tree. (So far so good). In the second component, ``we
assume the value of $Y_{i}$ evolves toward either 0 or 1 depending on the values
of [the species-specific independent variables], with the rate of evolution no
longer depending on the transition rate $\alpha$ but instead depending on the
regression coefficients $b_{j}$ for the independent variables $X^{j}$''.

Does this second step change the values of $Y$ obtained in the first step?

\section{2022-02-28: Hipp et al 2017: Sympatric parallel Diversification of
  major oak clades}

Used 300 samples from 146 species of oak trees using RAD-seq sequencing. An ML
phylogeny was inferred and analyzed alongside other types of data to reconstruct
the evolutionary history of American oaks.

We find a sympatric parallel diversification in niche, leaf habit, and
diversification rates. We find that oaks adapted rapidly to niche transitions.

Key part: This paper investigated the number of transitions in leaf habit for
the phylogeny as a whole. Leaf habit is treated as a binary trait. Phylogenetic
regression was used to assess the partial effects of various variables
(temperature seasonality, moisture stress, etc) on variance in leaf habit.
Analyses were performed with phylogenetic logistic regression, the technique
introduces bye today's other paper: Ives and Garland 2009.

Results for this: Mexican oak clades transition between deciduous and evergreen
habit twice as frequently as the remainder of the oaks. The phylogenetic
logistic regression model and the ``generalized estimating equations'' (a method
from a different paper) both point to two models as the ``best fit'': one in
which all the factors are included, and one in which only moisture stress and
mean temperature in the warmest quarter are included. They don't agress on which
is best, however. Don't know which method is better.

\section{Questions} What is a jackknife dataset?

I don't understand how to interpret bootstrap support (for example in figure 2
of Hipp 2017). As well as the boostrap confidence intervals and bootstrap p
values in Ives and Garland. 

What is parallel sympatric diversificaton?


\section{2022-03-03: Revell 2013: Two new graphical models for mapping trait
  evolution on phylogenies}

Two different graphical models are presented for visualizing phenotypic
evolution on the tree.
\begin{enumerate}
\item Method 1: a new approach for plotting the posterior density of
  stochastically mapped character histories for a binary (two-state) phenotypic
  trait on a phylogeny
\item Method 2: a closely related technique that uses ancestral character
  estimation to visualize historical character states for a continuous trait
  along the branches of the tree.
\end{enumerate}
The second method has the downside of ignoring the uncertainty about ancestral
traits. In this paper, a ``traitgram'' is presented which allows visualization
of ancestral state uncertainty.

An interesting equation from Felsenstein is used in both papers
\begin{equation*}
  \hat{a}= \frac
  {\frac{x_{i}}{v_{i}}+ \frac{x_{j}}{v_{j}}}
  {\frac{1}{v_{i}}+\frac{1}{v_{j}}}
\end{equation*}
to interpolate the value $\hat{a}$ of a trait on an edge by its values $x_{i}$
and $x_{j}$ at phylogenetic distances $v_{i}$ and $v_{j}$ away. This can be interpreted as a weighted harmonic mean (as on the wikipedia page for harmonic mean, with $w_{i}=x_{i}/v_{i}$). Is there a simple intuition why this is the appropriate way to interpolate character values? 

Also, was BUCKy was named after the badger?

\textbf{Stochastic character mapping} involves sampling possible histories of a
discretely valued character from their Bayesian posterior distribution. When you
have thousands of such character maps, you can estimate the history of a
discretely-valued character trait which evolved on the tree. Unfortunately, it's
hard to visualize this on a figure for publication.


\section{2022-03-03: Kriebel 2014: Discovery of unusual anotomical and
  continuous characters in the evolutionary history of Conostegia}

Conostegia is a genus of flowering plants native to the Americas. This paper
finds to to be paraphyletic. Unusual characters were documented and their
evolution studied in light of the phylogeny. The importance of documenting
anatomical and continuous characters is demonstrated.

Use of both binary and continuous characters. The continuous characters used were: herkogamy and seed size.

Two methods used to reconstruct ancestral states:
SIMMAP -- stochastic character mapping
ACE -- ancestral character estimation

\subsection{Phylogenetic Analyses}
MAFFT to make alignmnets. ML analyseses done using RAxML. Bayesian analyses done
on partitioned dataset using MrBayes. Convergence analyses using Tracer v1.5 and
AWTY. Bayesian concordance analyses (BCA) implemented in BUCKy,

\subsection{Presentation about Kriebel}

Researchers were interested in determining if the genus Conostegia is monophyletic, also how did the traits evolve and what are the relationships within Conostegia?

Sampled 31 species of Conostegia.

Used a combined dataset of both plastid and nuclear markers. 15 characters scored: 13 binary (e.g. calyx teeth present or absent) and 2 continuous (e.g. seed length). 

You can have discordances between different trees as a result of ILS, hybridization, etc. Bucky evaluates discordant trees.

There were some differences between concatenated vs concordance trees, suggesting the possibility of ancient hybridization or chloroplast capture.

\section{Does this get added with git?}
\end{document}
