\documentclass{article}

% Math Packages
\usepackage{amsmath}
\usepackage{amssymb}
\usepackage{amsthm}
\usepackage{amsfonts}
\usepackage{bbm}
\usepackage{breqn}
\usepackage[margin=1in]{geometry}
\usepackage{graphicx}
\usepackage{tikz}
\graphicspath{ {./images/} }

% Colorful Notes
\usepackage{color}
\definecolor{Red}{rgb}{1,0,0}
\definecolor{Blue}{rgb}{0,0,1}
\def\red{\color{Red}}
\def\blue{\color{Blue}}
\newcommand{\rnote}[1]{{\red(#1)}} % \rnote{foo} then 'foo' is red
\newcommand{\bnote}[1]{{\blue(#1)}} % \bnote{foo} then 'foo' is blue

% Concise Vector Macro
% Write bracket vectors of arbitrary length using commands like \cv{1,2,3}
% or \cv{0,1} ('cv' for 'column vector')
\usepackage{stackengine}
\newcommand\cv[1]{\setstackEOL{,,}\bracketVectorstack{#1}}

% Math Environments
\newtheorem{theorem}{Theorem}
\newtheorem{definition}{Definition}
\newtheorem{assumption}{Assumption}
\newtheorem{lemma}{Lemma}
\newtheorem{remark}{Remark}
\newtheorem{prop}{Proposition}
\newtheorem{corollary}{Corollary}
\newtheorem{example}{Example} 
\newtheorem{question}{Question}
\newtheorem{claim}{Claim} 
\newtheorem{conjecture}{Conjecture} 

%Custom Math Commands
\newcommand{\vt}{\vskip 5mm} % vertical space
\newcommand{\fl}{\noindent\textbf} % first line
\newcommand{\Fl}{\vt\noindent\textbf} % first line with space above
\newcommand{\norm}[1]{\left\lVert#1\right\rVert} % norm
\newcommand{\pnorm}[1]{\left\lVert#1\right\rVert_p} % p-norm
\newcommand{\qnorm}[1]{\left\lVert#1\right\rVert_q} % q-norm
\newcommand{\1}[1]{\textbf{1}_{\left[#1\right]}} % indicator function 
\newcommand{\EE}[1]{\mathbb{E}\left[#1\right]} % expectation with bracket
\newcommand{\PP}[1]{\mathbb{P}\left[#1\right]} % expectation with bracket
\def\limt{\lim_{t\to\infty}} % shortcut for lim as t-> infinity
\def\limn{\lim_{n\to\infty}} % shortcut for lim as n-> infinity
\def\sumn{\sum_{n=1}^{\infty}} % shortcut for sum from n=1 to infinity
\def\sumkn{\sum_{k=1}^{n}} % shortcut for sum from k=1 to n
\def\sumin{\sum_{i=1}^{n}} % shortcut for sum from i=1 to n
\def\SAs{\sigma\text{-algebras}} % shortcut for $\sigma$-algebras
\def\SA{\sigma\text{-algebra}} % shortcut for $\sigma$-algebra
\def\Fn{\mathcal{F}_n} % dicrete filtration
\def\Ft{\mathcal{F}_t} % continuous time filration (t)
\def\Fs{\mathcal{F}_s} % continuous time filtration (s)
\def\F{\mathcal{F}} % sigma-algebra
\def\G{\mathcal{G}} % sigma-algebra
\def\R{\mathbb{R}} % Real numbers
\def\Z{\mathbb{Z}} % Integers
\def\E{\mathbb{E}} % Expectation
\def\P{\mathbb{P}} % Probability

\def\fin{< \infty} % Is Finite
\def\a{\alpha} %alpha
\def\e{\epsilon} %epsilon
\def\th{\theta} %theta
\def\lb{\lambda} %lambda
\def\t{\tau} %tau

% Brackets and Parentheses
\def\[{\left [}
\def\]{\right ]}
\def\({\left (}
\def\){\right )}
\def\<{\langle}
\def\>{\rangle}


% Macros specific to this class
\newcommand{\advar}{\sigma_{a}^{2}} % additive variance
\newcommand{\bvar}{\sigma_{B}^{2}}  % between population variance
\newcommand{\mvar}{\sigma_{m}^{2}}  % mutational variance
\newcommand{\wvar}{\sigma_{w}^{2}}  % phenotypic variation within a population
\newcommand{\svar}{\sigma_{s}^{2}}  % per generation selection variance

\title{Lecture-notes}

\begin{document}



\section{Botany 940 Notes}
\subsection{2021-02-01: Brownian Motion}

\subsubsection{Hamon Chapter 3: Intro to Brownian Motion}

We consider traits that have continuous distribution --- such as body mass. We
can model these traits with brownian motion. It is incorrect to equate brownian
motion models with models of pure genetic drift. (Why?)

When modeling evolution using brownian motion, we usually are considering the
dynamics of the \textbf{mean character value} $\overline{z}$ in a population.
This is the average value of the trait within the population. We denote by
$\overline{z}(t)$ the value at time $t$. We can model this by a brownian motion
process. The mean is $\overline{z}(0)$ and the variance, or ``evolutionary rate
parameter'' is denoted by $\sigma^{2}$. We can simulate change under the
brownian motion model by drawing from normal distributions; simply put the
changes in a trait value over an interval of time $t$ is drawn from a normal
random variable with mean 0 and variance $\sigma^{2}t$.


What does this mean: ``We assume that mutations are drawn at random from a
distribution with mean 0 and mutational variance $\sigma_{m}^{2}$.''?
\begin{definition}[Types of Variance]
Types of variance under this model:
\begin{itemize}
  \item $\mvar$ is the \textbf{mutational variance} the variance of the number of
    new distinct mutations added per generations (I think)
  \item $\bvar$ is \textbf{between-population phenotypic variance}, meaning the
    variance of mean trait values across many independent runs of evolutionary
    change over a certain time period.
  \item $\advar$ is the \textbf{additive genetic variance} within each
    population at some time $t$. Additive genetic variance measures the total
    amount of genetic variation that acts additively (i.e. the contributions of
    each allele add together to predict the final phenotype)
  \item $\wvar$ is the total phenotypic variation within a population, including
    both non-additive genetic effects and environmental effects.
\end{itemize}
\end{definition}

\subsubsection{Additive variance equilibrium}
\begin{claim}
  We have $\E\[ \advar (t)\] \approx 2N_{e}\sigma_{m}^{2}$ for large $t$.
\end{claim}
\begin{proof}
  Over time, $\advar$ will change. This is due to genetic drift (which tends to
  decrease $\advar$) and mutational input (which tends to increase $\advar$). The
  expected value of $\advar$ changes from one generation to the next according to
  the formula:
  \begin{equation*} \E\[\advar(t+1) \]= \left( 1-\frac{1}{2N_{e}}
    \right)\E\[\sigma_{a}^{2}(t) \]+\sigma_{m}^{2}
  \end{equation*}
  where $t$ is the time in generations. The first term represents the loss due to
  drift. The second term represents the gain due to new mutations each generation.
  By induction, we have:

  \begin{equation}\label{eq:2}
    \E\[\advar(t) \] = \left( 1-\frac{1}{2N_{e}} \right)^{t}\left[ \advar(0) -2N_{e}\mvar \right] + 2N_{e}\mvar 
  \end{equation}
  where of course $\advar(0)$ is the starting value at time zero. This formula
  allows us to calculate the expected additive genetic variance at any time
  provided we know $\advar(0)$. Since $\left( 1-\frac{1}{2N_{e}} \right)^{t}\to 0$
  as $t\to\infty$, we expect $\E\[ \advar (t) \]$ to approach equilibrium value of
  $2N_{e}\sigma_{m}^{2}$.
\end{proof}


\subsubsection{Derivation of $\bvar(t)$}
  

Assume $\advar$ is at equilibrium and thus constant. Harmon claims this is
justified by equation \eqref{eq:2}, even though that equation is about the
expected value of $\advar$. Then skipping some calculations,
\begin{equation}
  \label{eq:1}
  \bvar(t)= \frac{t\advar(t)}{N_{e}}
\end{equation}

Substituting the equilibrium value $2N_{e}\mvar$ in for $\advar(t)$ gives
\begin{equation}
  \label{eq:4}
  \bvar(t) = 2t \mvar
\end{equation}

This says that the variation between two diverging populations depends on twice
the time since they diverged and the rate of mutational input. No dependence on
population size or starting state is observed, so under this model, long-term
rates of evolution are dominated by the supply of new mutations to a population.

Actually \eqref{eq:4} is very general, holding under a range of models.
Unfortunately, we cannot measure $\mvar$ in any natural population.




\begin{definition}[Heritability]
  \textbf{Heritability} is the proportion of total phenotypic variation $\wvar$
  within a population that is due to additive genetic effects; namely,
  \begin{equation}
    \label{eq:5}
    h^{2}=\frac{\advar}{\wvar}.
  \end{equation}
\end{definition}

By the additive variance equilibrium equation \eqref{eq:2}, we have
\begin{equation}
  \label{eq:6}
  h^{2}= \frac{2N_{e}\mvar}{\wvar}
\end{equation}
or equivalently,
\begin{equation}
  \label{eq:7}
  \mvar = \frac{h^{2}\wvar}{2N_{e}}
\end{equation}
and therefore by \eqref{eq:4},
\begin{equation}
  \label{eq:8}
  \bvar(t) = \frac{h^{2}\wvar t}{N_{e}}.
\end{equation}


\subsection{Brownian Motion under Selection}
Recall we with to model the path followed by population mean trait values under
mutation, selection, and drift. The brownian motion model applies even to some
cases with selection.

\subsubsection{Random Directional Selection}
We assume directional selection, but with the strength and direction of
selection varying randomly from one generation to the next. Each generation, we
model selection as being drawn from a normal distribution with mean 0 and
variance $\svar$.

In this case, for some reason, we have
\begin{equation}
  \label{eq:9}
  \bvar = \left( \frac{h^{2}\wvar}{N_{e}}+ \svar \right)t
\end{equation}
In this case, if variation in selection is much greater than varition due to
drive, then for some reason we have
\begin{equation}
  \label{eq:3}
  \bvar \approx \svar
\end{equation}

\subsubsection{Random Stabilizing Selection}

On the other hand, if the trait is under stabilizing selection for a particular
optimal value, where the position of the optimal value changes randomly
according to a brownian motion process with variance $\sigma_{E}^{2}$, then
\begin{equation*}
  \bvar \approx \sigma^{2}_{E}
\end{equation*}

Under both of these models, the pattern of trait evolution thorugh time still
follows a brownian motion model, even though the changes are dominated by
selection rather than drift. Therefore brownian motion evolution does not assume
that characters are not under selection.

\subsubsection{Constant Directional Selection}

Not going to go through the equations on this one. If a trait is under
directional selection of strength $s$ in both populations, then we get
\begin{equation}
  \label{eq:10}
  \bvar(t) = \frac{h^{2}\wvar t}{N_{e}}
\end{equation}
this does not depend on $s$ and in fact is the same equation as \eqref{eq:8}
which was derived earlier. Thus we can't tell by looking at two living
populations whether the trait we are measuring was subject to selection, Or
something like that.

\subsection{Brownian Motion on a Tree}
I am mostly familiar with this section from the diffusion group. A new thing is
the \textbf{phylogenetic variance-covariance matrix} \textbf{C}. This matrix has a special
structure. For phylogenetic trees with $n$ species, this is an $n\times n$
matrix, with each row and column corresponding to one of the $n$ taxa in the tree.
Along the diagonal are the total distances of each taxon from the root of the
tree, while the off-diagonal elements are the total branch lengths shared by
particular pairs of taxa. So it is a matrix $(\sigma^{2}x_{ij})$ where $x_{ii}$
is the length of the path $P(i,r)$ from leaf $i$ to the root, and $x_{ij}$ is
the length of those edges shared by paths $P(i,r)$ and $P(j,r)$.

Example: suppose we have a phylogenetic tree with two leaves $1$ and $2$ with
lengths $t_{2},t_{3}$ respectively, and a root edge of length $t_{1}$. We can
describe the trait values for the two species as a single draw from a
multivariate normal distribution. Each trait has the same expected value
$\overline{z}(0)$, which is the starting value of the trait at the root, and two
traits have the following covariance matrix:

\begin{equation*}
  \begin{bmatrix}
    \sigma^2 (t_1 + t_2) & \sigma^2 t_1 \\
    \sigma^2 t_1 & \sigma^2 (t_1 + t_3) \\
\end{bmatrix}
= \sigma^2
\begin{bmatrix}
    t_1 + t_2 & t_1 \\
    t_1 & t_1 + t_3 \\
\end{bmatrix} = \sigma^2 \mathbf{C}
\end{equation*}
we see what $\mathbf{C}$ is in this case.

\section{Fitting Brownian Motion Models to Single Characters}
\subsection{Log Transforms}
If your data is a continuous trait and you think percentage changes matter more
than absolute changes, then you should do a log-transform of your data.

\subsection{Estimating Rates Using Independent Contrasts}

\textbf{Phylogenetical idependent contrasts} (PICs) are a way to estimate the
rate of character change across a phylogeny. There is an algorithm, the
algorithm of independent contrats, from Felsenstein. It is used to estimate the
rate of evolution under a Brownian model. It uses some of the following ideas.
The \textbf{raw contrast} is the difference between the value of the character
at two chosen tips: $c_{ij} = x_{i}-x_{j}$. Under the brownian model, $c_{ij}$
has expectation zero and variance proportional to $v_{i}+v_{j}$. The
\textbf{standardized contrast} is $s_{ij} = \frac{c_{ij}}{v_{i}+v_{j}}$. It
follows that $s_{ij}\sim N(0,\sigma^{2})$.

\subsection{Estimating Rates Using Maximum Likelihood}
Another way to estimate evolutionary rate is by finding the maximum-likelihood
parameter values for a brownian motional model. Our data is character values at
the tips of the tree which we assume was generated with the Brownian model.
There is a standard formula for the likelihood of drawing from a multivariate
normal distribution; remember we wish to find the parameter values that maximize
this function; this is done using optimization algorithms.

Actually, in some cases we have analytic solutions.

\textbf{Restrictied Maximum Liklihood} (REML) refers to an approach which uses
ML after transforming the data set to remove nuisance paramters, such as the
root state.


\subsection{Bayesian Approach to Evolutionary Rates}
This approach uses explicit priors for parameter values, and then runs an MCMC
to estimate posterior distributions of parameter estimates.


\section{Fitting Brownian Models to Multiple Characters}
Motivation: A wide variety of hypotheses can be framed as tests of correlations
between continuously varying traits across species. For example, is the body
size of a species related to its metabolic rate? How does the head length of a
species relate to overall size, and do deviations from this relationship relate
to an animal’s diet?


We make a distinction between \textbf{standard correlation} and
\textbf{evolutionary correlation}. the first means that one trait predicts the
value of another. The second is when two traits evolve together due to a process
like mutation, drift, or selection. If there is an evolutionary correlation
between two characters, that means we can predict the magnitude and direction of
changes in one character given knowledge of evolutionary changes in another.

Phylogenetic relatedness alone can lead to a relationship between two variables
that are not in face evolving together. In that case, there is a confounding
variable -- the clade the related species belong to.

\subsection{Modeling the Evolution of Correlated Characters}
Multivariate brownian motion. Each trait evolves under a brownian motion, but
these brownian motions are not necessarily independent, as specified by some
covariance matrix $\mathbf{R}$.

Recall that correlations at the tips depend also on branch lengths, specified by
some matrix $C$ (see chapter 3). Then the variance-covariance matrix is the
Kroeneker product $\mathbf{R}\otimes \mathbf{C}$, an $nr\times nr$ matrix, where
$n$ is the number of leaves of the tree and $r$ the number of traits.

Here again, there is a formula for the likelihood of observing data
$\mathbf{x}_{nr}$ given the input parameters. Also, we can estimate these
parameters analytically with formulas similar to the univariate case.

\subsection{Testing for Evolutionary Correlations}
Fit a model with correlation. Fit a model without correlation. Calculate the
optimal log-likelihood in both cases. Apply log-likelihood test.

There is also a bayesian approach that I didn't read.

\subsection{Testing with Traditional Approaches (PIC, PGLS)}

PGLS is \textbf{Phylogenetic General Least Squares}. In general least squares,
we construct a model of the relationship between the column vectors
$x,y\in \R^{n}$ of trait values and the whose correlation we wish to test:
\begin{equation*}
  y = X_{D}b + \epsilon
\end{equation*}
where
$X_{D} = \begin{bmatrix} 1 & x_1 \\ 1 & x_2 \\ \vdots & \vdots \\ 1 & x_n
  \\ \end{bmatrix}$, $b$ is $2\times 1$, and the error vector $\epsilon$ is
assumed to be multivariate normal with mean zero and some covariance matrix
$\Omega$. In the Brownian model, we assume that the residuals have variances and
covariances following the structure of the phylogenetic tree: that is, given by
the matrix $\mathbf{C}$. We carry out standard least square to estimate model
parameters. The first term in $b$ is the phylogenetic mean $\overline{z}(0)$,
and the second term is an estimate for the slope of the relationship between $y$
and $x$.

Another way to think about a PGLS model is that we are treating $x$ as a fixed
property of species. The deviation of $y$ from what is predicted by $x$ is what
evolves under a Brownian motion model.


\end{document}